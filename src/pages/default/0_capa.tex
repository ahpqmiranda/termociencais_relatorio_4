\thispagestyle{empty}

\begin{center}
	\begin{figure}[h]
		\centering
		\includegraphics[width=0.21\linewidth]{logos/Brasao-UFPA-colorido}
		\label{fig:ufpa}
	\end{figure}

	\large \uppercase{UNIVERSIDADE FEDERAL DO PARÁ}\\
	\large \uppercase{INSTITUTO DE TECNOLOGIA}\\
	\large \uppercase{FACULDADE DE ENGENHARIA MECÂNICA}\\
	\vspace{7cm}

	\large \uppercase{INSTRUMENTAÇÃO EM TERMOCIÊNCIAS} \\ % << disciplina
	\vspace{1cm}

	\large \uppercase{DETERMINAÇÃO DE PERFIL DE VELOCIDADE E CÁLCULO DA
VELOCIDADE MÉDIA DE ESCOAMENTO} \\
	\vspace{7cm}

	\large BELÉM/PA \\ \the\year
\end{center}

\newpage
\thispagestyle{empty}
\begin{center}
	\large  \uppercase{ALAN HENRIQUE PEREIRA MIRANDA - 202102140072} \\
	\large \uppercase{Alexandro Aldo Lopes Osório – 202302140096} \\
	\large \uppercase{ANA MONICA CARDOSO DA COSTA - 202202140017} \\
	\large \uppercase{HABLIEL FELIX DE CARVALHO - 202202140073} \\


	\vspace{1cm}
	\large INSTRUMENTAÇÃO EM TERMOCIÊNCIAS \\
	\vspace{1cm}
	\large \uppercase{DETERMINAÇÃO DE PERFIL DE VELOCIDADE E CÁLCULO DA
VELOCIDADE MÉDIA DE ESCOAMENTO} \\
\end{center}
\vspace{7cm}

\singlespacing
\hspace{8cm}
\begin{minipage}{7cm}
	Atividade apresentada à disciplina de Instrumentação em Termociências, ministrada pelo professor Marcelo de Oliveira E Silva, para aprovação disciplinar.\\

	\vspace{0.5cm}
	Prof. Dr. Marcelo de Oliveira E Silva\\

	\vspace{1cm}
	Belém-PA, 29 de Janeiro de 2026.
\end{minipage}

	\vspace{1cm}
\onehalfspacing
\begin{center}
	EXAMINADOR\\
	\vspace{3cm}
	\rule{10cm}{0.15mm} \\
	Prof. Dr. Marcelo de Oliveira E Silva\\
	Universidade Federal do Pará - UFPA
\end{center}

\newpage
\thispagestyle{empty}
\begin{center}
	\listoffigures
\end{center}

\newpage
\thispagestyle{empty}
\begin{center}
	\tableofcontents
\end{center}
\newpage
