\section{Introdução}
\label{sec:introducao}

O escoamento de fluidos em dutos é fundamental para caracterizar perfis de velocidade que variam radialmente devido a efeitos de viscosidade na parede.
Assim sendo, este experimento utiliza um ventilador com duto equipado com tomadas de pressão estática, tubo de Pitot e manômetro em U para medir velocidades locais em função do raio, aplicando a equação de Bernoulli para converter diferenças de pressão em velocidades.

A base teórica envolve a conservação de massa para integrar o perfil e obter a velocidade média, consolidando conceitos de mecânica dos fluidos em regime permanente \cite{tcc_kallyne, conservacao_massa_unip}.
Tais ensaios são base para a validação empírica da equação da continuidade e para a caracterização de instrumentação de escoamento \cite{equacao_continuidade_brasilescola, instrumentacao_vazao}.

\section{Objetivos}
\label{sec:objetivos}

Determinar experimentalmente o perfil de velocidade de escoamento em um duto e calcular a velocidade média aplicando a conservação de massa \cite{experiencia04, mecflu_ufpr}.

\subsection{Objetivo Geral}
\label{subsec:objetivo_geral}
Determinar o perfil de velocidade e a velocidade média de escoamento em um duto utilizando medições de pressão estática e dinâmica através da equação de Bernoulli \cite{equacao_bernoulli_youtube, medida_velocidade_unip}.

\subsection{Objetivos Específicos}
\label{subsec:objetivo_especificos}
\begin{itemize}
    \item Medir a pressão estática e dinâmica em diferentes pontos radiais do duto utilizando a sonda de Pitot \cite{tcc_rodrigo_ufsc}.
    \item Calcular a velocidade local utilizando a equação derivada de Bernoulli \cite{equacao_bernoulli_youtube}.
    \item Integrar o perfil de velocidade para obter a velocidade média \cite{experiencia04}.
\end{itemize}



\section{Metodologia}
\label{sec:metodologia}

O aparato experimental é composto por um túnel de vento confeccionado em tubo de PVC de 6 cm de diâmetro, acoplado a um ventilador industrial responsável por gerar o escoamento de ar. Para aferir o perfil de velocidades, utiliza-se um tubo de Pitot que se desloca radialmente no interior do duto \cite{experiencia04, tcc_kallyne}. O uso do Pitot neste formato é uma prática clássica documentada em trabalhos experimentais para determinação do perfil \cite{perfil_velocidades_scribd}.

A inovação na instrumentação consiste no uso de um sensor de pressão diferencial conectado a um microcontrolador Arduino, prática que vem ganhando espaço em adaptações modernas de túneis de vento \cite{medindo_velocidade_vento, trabalho_phoenics}. O sensor capta a pressão de estagnação do ar e a converte em um sinal de tensão (0 a 5V). O conversor analógico-digital (ADC) de 10 bits do Arduino traduz essa tensão em uma leitura bruta (\textit{raw}), gerando valores inteiros que variam de 0 a 1023. Simultaneamente, utiliza-se um manômetro em U clássico, com coluna de água em milímetros (mm), para fornecer o referencial físico indispensável para a calibração do transdutor eletrônico \cite{experiencia04, estudo_pitot_lamon}.

\subsection{Procedimento Experimental}
\label{subsec:procedimento_experimental}

O experimento foi conduzido sob duas condições de vazão do ventilador: a 100\% (capacidade total) e a 60\%. Em cada etapa, o tubo de Pitot foi transladado da parede interna esquerda à direita do tubo de PVC (0 cm a 6 cm, com espaçamento de 1 cm entre medições) \cite{perfil_velocidade_youtube}. Em cada uma das 7 posições de perfil, anotou-se simultaneamente a leitura bruta do canal analógico do Arduino e o desnível real da coluna de água \cite{experiencia04}.

\subsection{Tratamento Computacional}
\label{subsec:tratamento_computacional}

A transformação da leitura digital para a velocidade do escoamento exige três passos lógicos, que foram estruturados através de um script em Python utilizando as bibliotecas \texttt{numpy}, \texttt{pandas} e \texttt{scipy}.


A transformação da leitura digital para a velocidade do escoamento exige três passos lógicos, que foram estruturados através de um script em Python utilizando as bibliotecas \texttt{numpy}, \texttt{pandas} e \texttt{scipy}.

\begin{enumerate}
    \item \textbf{Calibração do Sinal (Regressão Linear):}
    Em vez de utilizar a função de transferência de fábrica do sensor, os dados brutos do Arduino (0 a 1023) foram correlacionados com as medições empíricas da coluna de água (\text{mm}). Aplicou-se uma regressão linear sobre os dados unidos das duas vazões para encontrar uma função de calibração do tipo: \(h_{mm} = A \cdot \text{ADC} + B\). O Python retornou a seguinte equação empírica:
    \[ h_{mm} = 1.5454 \cdot \text{ADC} - 808.9184 \]

    \item \textbf{Conversão de Unidade Dimensional:}
    A coluna de água, aferida em milímetros, é convertida para metros dividindo-se o valor por 1000 (\(h_m = \frac{h_{mm}}{1000}\)), garantindo a compatibilidade com o Sistema Internacional de Unidades (SI).

    \item \textbf{Validação da Teoria (Equação de Bernoulli):}
    Com o diferencial de pressão mapeado fisicamente em metros de coluna de água, aplica-se a equação teórica derivada de Bernoulli para extrair a velocidade local do escoamento (\(v\)):
    \[ v = \sqrt{\frac{2 \cdot \rho_{agua} \cdot g \cdot h_m}{\rho_{ar}}} \]
    Onde adota-se a densidade da água (\(\rho_{agua}\)) como \(1000 \, \text{kg/m}^3\), a densidade do ar (\(\rho_{ar}\)) como \(1,225 \, \text{kg/m}^3\) e a aceleração da gravidade (\(g\)) como \(9,81 \, \text{m/s}^2\).
\end{enumerate}

\subsection{Código de Estruturação em Python}
\label{subsec:codigo_python}

Abaixo está o bloco de código desenvolvido para o processamento automatizado, que integra a leitura dos dados planilhados, o cálculo da calibração global e a aplicação da equação de Bernoulli.

\begin{lstlisting}[language=Python]
import numpy as np
import pandas as pd
from scipy.stats import linregress

# 1. Carregando os dados do experimento a partir da planilha
data100_df = pd.read_excel(io='dados exp4.ods', sheet_name='abertura_100', engine='odf')
data60_df = pd.read_excel(io='dados exp4.ods', sheet_name='abertura_60', engine='odf')

# 2. Unindo os dados para a Regressão Linear Global
todas_medicoes = np.concatenate([data100_df['medição'].values, data60_df['medição'].values])
todas_colunas = np.concatenate([data100_df['coluna mm'].values, data60_df['coluna mm'].values])

slope, intercept, _, _, _ = linregress(todas_medicoes, todas_colunas)
# Equação resultante: h(mm) = slope * ADC + intercept

# 3. Constantes Físicas para a Equação de Bernoulli
g = 9.81           # m/s^2
rho_agua = 1000    # kg/m^3
rho_ar = 1.225     # kg/m^3

# 4. Função de conversão automatizada: ADC -> mmH2O -> Metros -> m/s
def adc_para_velocidade(leitura_adc):
    h_mm = (slope * leitura_adc) + intercept    # Calibração via Regressão
    h_m = h_mm / 1000.0                         # Conversão mm para m

    # Tratamento para evitar raízes negativas por flutuação do sensor
    if h_m < 0:
        h_m = 0

    velocidade = np.sqrt((2 * rho_agua * g * h_m) / rho_ar) # Bernoulli
    return velocidade

# 5. Aplicação da função nos DataFrames
data100_df['velocidade (m/s)'] = data100_df['medição'].apply(adc_para_velocidade)
data60_df['velocidade (m/s)'] = data60_df['medição'].apply(adc_para_velocidade)
\end{lstlisting}



\section{Resultados e Discussões}
\label{sec:resultados}

Esta seção apresenta os resultados obtidos após o tratamento computacional dos dados aferidos no túnel de vento, dividindo-se na validação da instrumentação eletrônica e na análise física do perfil de velocidades do fluido.

\subsection{Validação e Calibração do Sensor}
\label{subsec:calibracao}

A primeira etapa do tratamento consistiu em correlacionar a leitura bruta do conversor analógico-digital (ADC) do Arduino com a diferença de pressão real, medida em milímetros de coluna de água. A Figura \ref{fig:calibracao_sensor} apresenta a curva de calibração gerada pela regressão linear dos dados empíricos de ambas as vazões (100\% e 60\%).

\begin{figure}[!htbp]
    \centering
    \caption{Curva de Calibração do Sensor de Pressão (ADC vs. Coluna de Água).}
    \includegraphics[width=0.8\textwidth]{scripts/grafico1_experimento_pitot.png}
    \label{fig:calibracao_sensor}

    \text{Fonte: Elaborado pelos autores (2026)}
\end{figure}

O modelo linear gerado pelo algoritmo demonstrou a seguinte relação matemática:
\[ h_{mm} = 1.5454 \cdot \text{ADC} - 808.9184 \]

O coeficiente de determinação ($R^2 = 0.8513$) indica uma correlação linear satisfatória entre o sinal elétrico e a pressão física. Embora o valor de $R^2$ evidencie uma leve dispersão típica de sensores analógicos de baixo custo operando em faixas sensíveis de pressão, o comportamento linear foi suficientemente robusto para validar o uso do Arduino na substituição da leitura visual da coluna de água \cite{trabalho_phoenics}.

\subsection{Perfil de Velocidades do Escoamento}
\label{subsec:perfil_velocidades}

Com a calibração validada, a equação de Bernoulli foi aplicada para converter os diferenciais de pressão em velocidades locais do fluido (m/s) \cite{medida_velocidade_unip}. Os resultados calculados para as configurações de 100\% e 60\% de abertura do ventilador estão consolidados, respectivamente, na Tabela \ref{tab:pitot_100} e na Tabela \ref{tab:pitot_60}.

% Importação das tabelas geradas pelo seu código Python
\begin{table}[!htbp]
    \centering
    \caption{Velocidades Calculadas - Abertura 100\%}
    \label{tab:pitot_100}
    \begin{tabular}{|c|c|c|c|c|c|}
        \toprule
        \hline
        ponto   & raw arduino & coluna mm & abertura & posição (cm) & velocidade (m/s) \\
        \hline
        \midrule
        \hline
        \hline1 & 545.13  & 44        & 1.00     & 0            & 23.17            \\
        \hline2 & 557.38  & 58        & 1.00     & 1            & 28.98            \\
        \hline3 & 563.57  & 64        & 1.00     & 2            & 31.51            \\
        \hline4 & 567.55  & 69        & 1.00     & 3            & 33.04            \\
        \hline5 & 566.27  & 67        & 1.00     & 4            & 32.56            \\
        \hline6 & 563.04  & 62        & 1.00     & 5            & 31.31            \\
        \hline7 & 556.52  & 56        & 1.00     & 6            & 28.61            \\
        \hline
        \bottomrule
    \end{tabular}


    \text{Fonte: Elaborado pelos autores (2026)}
\end{table}

\begin{table}[!htbp]
    \centering
    \caption{Velocidades Calculadas - Abertura 60\%}
    \label{tab:pitot_60}
    \begin{tabular}{|c|c|c|c|c|c|}
        \toprule
        \hline
        ponto   & raw arduino & coluna mm & abertura & posição (cm) & velocidade (m/s) \\
        \hline
        \midrule
        \hline
        \hline1 & 540.22  & 20        & 0.60     & 0            & 20.38            \\
        \hline2 & 543.63  & 24        & 0.60     & 1            & 22.35            \\
        \hline3 & 548.20  & 28        & 0.60     & 2            & 24.75            \\
        \hline4 & 549.20  & 30        & 0.60     & 3            & 25.25            \\
        \hline5 & 547.13  & 30        & 0.60     & 4            & 24.21            \\
        \hline6 & 540.15  & 28        & 0.60     & 5            & 20.33            \\
        \hline7 & 532.38  & 26        & 0.60     & 6            & 14.87            \\
        \hline
        \bottomrule
    \end{tabular}

    \text{Fonte: Elaborado pelos autores (2026)}
\end{table}


A distribuição espacial dessas velocidades ao longo do diâmetro do tubo de PVC (de 0 a 6 cm) pode ser visualizada na Figura \ref{fig:perfil_velocidade}.

\begin{figure}[!htbp]
    \centering
    \caption{Perfil de Velocidade Radial no Duto de PVC.}
    \includegraphics[width=0.8\textwidth]{scripts/grafico2_experimento_pitot.png}
    \label{fig:perfil_velocidade}

    \text{Fonte: Elaborado pelos autores (2026)}
\end{figure}

A análise da Figura \ref{fig:perfil_velocidade} comprova o comportamento teórico esperado para o escoamento interno de fluidos viscosos em dutos cilíndricos \cite{tcc_kallyne, perfil_velocidade_youtube}. Observa-se a formação de um perfil parabólico, onde a velocidade atinge seu valor máximo na região central do tubo (posição 3 cm), registrando aproximadamente 33,2 m/s na capacidade máxima e 21,9 m/s na capacidade reduzida.

Nas proximidades das bordas do duto (posições 0 cm e 6 cm), nota-se uma queda acentuada na velocidade. Este fenômeno ocorre devido à tensão de cisalhamento gerada pelo atrito entre as partículas de ar e a parede interna do tubo de PVC, fenômeno associado à condição teórica de não-escorregamento amplamente discutida na literatura de mecânica dos fluidos \cite{velocidade_duto_scribd, mecflu_ufpr}. Além disso, a separação visual clara entre as curvas corrobora a consistência do experimento, provando que a redução da vazão do ventilador para 60\% refletiu em uma queda homogênea e proporcional do campo de velocidades em toda a seção transversal.

\section{Conclusão}
\label{sec:conclusao}

O presente relatório alcançou seu objetivo principal ao determinar experimentalmente o perfil de velocidades do escoamento de ar em um duto cilíndrico, validando os princípios teóricos da mecânica dos fluidos \cite{experiencia04, mecflu_ufpr}. A utilização combinada de um tubo de Pitot e um manômetro em U permitiu quantificar as pressões dinâmicas locais, que, através da Equação de Bernoulli, foram convertidas com sucesso em velocidades lineares \cite{equacao_bernoulli_youtube}.

A instrumentação do experimento demonstrou ser uma inovação eficaz \cite{medindo_velocidade_vento}. A substituição da leitura exclusivamente visual da coluna de água pelo emprego de um sensor de pressão diferencial acoplado a um microcontrolador Arduino possibilitou a automatização da aquisição de dados. O tratamento computacional desenvolvido em Python, fundamentado em uma regressão linear global ($R^2 = 0,8513$), garantiu uma calibração confiável do sinal analógico, viabilizando uma conversão de dados rápida, reprodutível e adequada ao Sistema Internacional de Unidades.

A análise gráfica dos resultados confirmou o comportamento fluido-dinâmico previsto pela literatura: o desenvolvimento de um perfil de velocidades em formato parabólico \cite{tcc_kallyne}. Ficou evidenciado que a velocidade atinge seu valor máximo no centro do tubo de PVC (aproximadamente 33,2 m/s na vazão total e 21,9 m/s na vazão de 60\%) e decai significativamente nas proximidades das paredes (0 cm e 6 cm). Esse decaimento corrobora o conceito físico de tensão de cisalhamento e atrito viscoso na condição de não-escorregamento \cite{velocidade_duto_scribd}.

Por fim, o aparato experimental e a metodologia computacional aplicada provaram-se métodos consistentes não apenas para o estudo acadêmico de perfis de velocidade, mas também como base sólida para futuras práticas de cálculo de vazão e velocidade média de escoamento em sistemas de ventilação \cite{instrumentacao_vazao}.
